\chapter{Local Materials List}
\label{cha:local-materials}

In order to gain a thorough understanding of science, students must be able to make a connection between classroom learning and the outside world. The application of scientific concepts to daily life is a critical component of a student's science education. The following is a list of locally available materials which may be used to substitute conventional materials and apparatus for various activities. These materials have the following advantages: 
\begin{itemize*}
\item They are readily available in the village or a nearby town;
\item They are cheaper than conventional materials; 
\item They may safely substitute the conventional materials without fear of losing accuracy or understanding; 
\item They help students to draw a connection between science education and the world around them.
\end{itemize*}
Imagination and innovativeness is encouraged on the part of the student and teacher to find other suitable local substitutions. \\

\noindent Throughout this book you will see materials that have been marked with an asterisk (*). These are locally available materials which can be made or purchased for your laboratory or classroom. The guide for using and making these local materials is found in this section.  

%\section{}
%\vspace{-10pt}
%\textbf{Use:} \\
%\textbf{Materials:} \\
%\textbf{Procedure:} 

