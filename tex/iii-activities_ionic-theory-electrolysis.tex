\section{Ionic Theory and Electrolysis} \index{Ionic theory}

\begin{multicols}{2}

% Taken from Extraction and Properties of Metals, Form III

\subsection{Displacement of Copper} \index{Displacement reactions}

%\begin{center}
%\includegraphics[width=0.4\textwidth]{./img/.jpg}
%\end{center}

\begin{description*}
%\item[Subtopic:]{}
\item[Materials:]{Steel wool, copper (II) sulphate, water, bottle}
\item[Setup:]{Prepare a copper (II) sulphate solution by dissolving a spoonful of crystals in about 500 mL of water.}
\item[Procedure:]{Pour 50 mL of the copper (II) sulphate solution into the container. Dip the steel wool into the solution and observe what happens.}
%\item[Hazards:]{}
%\item[Questions:]{}
\item[Observations:]{A layer of brown copper metal forms on the surface of the steel wool (this is not rust).}
\item[Theory:]{
%This is an example of a displacement reaction - metal ions in the solution will reduce and metal solids will oxidize if the proper combination is used. Oxidized compounds lose electrons, while reduced compounds gain electrons. The iron metal oxidizes to form iron (II) ions and the copper (II) ions reduce to form copper metal.

Metals can be arranged according to their reactivity, i.e. how likely they are to form positive ions. A metal higher in the reactivity series will displace a lower metal from a solution. 
\begin{center}
K $>$ Na $>$ Ca $>$ Mg $>$ Zn $>$ Fe $>$ Pb $>$ Sn $>$ H $>$ Cu $>$ Ag $>$ Au
\end{center}
Iron is higher than copper on the reactivity series, meaning iron ions displace copper ions in solution and the copper ions are deposited as copper metal. $$ \mathrm{CuSO}_{4(aq)} + \mathrm{Fe}_{(s)} \longrightarrow \mathrm{FeSO}_{4(aq)} + \mathrm{Cu(s)} $$}
%\item[Applications:]{}
%\item[Notes:]{}
\end{description*}

\subsection{Reactivity Rates Analogies} \index{Reactivity series}

\begin{center}
\includegraphics[width=0.4\textwidth]{./img/source/reactivity-race.jpg}
\end{center}

%\begin{description*}
%%\item[Subtopic:]{}
%\item[Materials:]{}
%\item[Setup:]{}
%\item[Procedure:]{}
%\item[Hazards:]{}
%\item[Questions:]{}
%\item[Observations:]{}
%\item[Theory:]{}
%\item[Applications:]{}
%\item[Notes:]{}
%\end{description*}

\vfill
\columnbreak

%\subsection{Displacement Reaction - Metal Reactivity} % Same as Displacement of Copper
%
%%\begin{center}
%%\includegraphics[width=0.4\textwidth]{./img/.jpg}
%%\end{center}
%
%\begin{description*}
%%\item[Subtopic:]{}
%\item[Materials:]{Nail, bottle, copper (II) sulphate}
%%\item[Setup:]{}
%\item[Procedure:]{Place the nail at the bottom of a container and cover it with copper (II) sulphate.}
%%\item[Hazards:]{}
%%\item[Questions:]{}
%\item[Observations:]{In 5 minutes a reddish brown precipitate of copper will form.}
%\item[Theory:]{This is an example of a displacement reaction - metal ions in the solution will reduce and metal solids will oxidize if the proper combination is used. Oxidized compounds lose electrons, while reduced compounds gain electrons. The iron metal oxidizes to form iron (II) ions and the copper (II) ions reduce to form copper metal, precipitating on the nail (this is not rust).}
%%\item[Applications:]{}
%\item[Notes:]{Repeat the activity using magnesium sulphate in place copper (II) sulphate. There will be no precipitate this time because the combination of compounds does not yield displacement.}
%\end{description*}

\subsection{Reactivity Series of Metals} \index{Metals! reactivity of}

\begin{center}
\includegraphics[width=0.49\textwidth]{./img/vso/reactivity-series.jpg}
\end{center}

\begin{description*}
%\item[Subtopic:]{}
\item[Materials:]{Glass sheet, large sheet of paper, metals, solutions of metal ions (see below)}
\item[Setup:]{Gather and clean small pieces of various metals (e.g.copper wire, iron wool, magnesium ribbon, zinc plate from dry cell, lead shot). Gather some solutions containing metal ions (e.g. copper (II) sulphate, iron (II) sulphate, magnesium sulphate, zinc sulphate, lead nitrate).}
\item[Procedure:]{Make a grid on the paper as shown. Place the glass sheet over the paper grid. Place the metals on the appropriate squares. Add 2-3 drops of a solution to each metal and observe and change.}
%\item[Hazards:]{}
%\item[Questions:]{}
\item[Observations:]{On some of the squares a black or red (in the
case of copper displacement) coating is formed
on the surface of the metal.}
\item[Theory:]{If a black coating forms on the metal it indicates that the metal ions are being displaced from the solution and deposited onto the metal. This shows that the metal is more reactive than the ion in solution. For example if Fe$^{2+}$ ions in
solution are dropped onto magnesium, the
magnesium displaces the Fe$^{2+}$ ions and a black
coating of iron can be seen on the surface of the
magnesium.}
%\item[Applications:]{}
%\item[Notes:]{}
\end{description*}

\columnbreak

%==================================================================================================%

\section*{Electrolytes and \hfill \\ Non-Electrolytes} \index{Electrolytes}


\subsection{Electrolytes}

\begin{center}
\includegraphics[width=0.49\textwidth]{./img/source/electrolytes.jpg}
\end{center}

\begin{description*}
%\item[Subtopic:]{}
\item[Materials:]{Table salt, distilled water, sugar, vinegar/citric acid, sulphuric acid, bottles, dry cells, clothes pegs, bulb/ammeter, speaker wire, 2 carbon electrodes (from dry cell)}
\item[Setup:]{Connect the circuit in series as shown. Take carbon electrodes from an old dry cell.}
\item[Procedure:]{Place the two carbon electrodes in a container of water. On a piece of paper, pour out some table salt crystals and touch the electrodes to them. Dissolve the salt in the water and again test for conductivity. Repeat with citric acid crystals and solution, then sugar crystals and solution, rinsing the electrodes between tests. Finally test the electrodes in a dilute sulphuric acid solution, rinse, then in vinegar.}
%\item[Hazards:]{}
%\item[Questions:]{}
%\item[Observations:]}
\item[Theory:]{Bubbles of gas at the electrodes indicate the
flow of an electric current. Substances which
conduct electricity in solution are called \emph{electrolytes}. No substance will conduct electricity in the solid state but some of them will
conduct in the dissolved state. Sodium chloride (table salt) is a \emph{strong electrolyte} - the bulb burns brightly. Vinegar and citric acid are \emph{weak electrolyte}s - the bulb burns dimly. Sugar is a \emph{non-electrolyte} - the bulb does not light. The bulb is brighter in sulphuric acid solution than in vinegar even though they are about the same concentration, because the sulphuric acid dissociates completely while citric acid/vinegar only partially dissociate into ions.}
%\item[Applications:]{}
%\item[Notes:]{}
\end{description*}

\subsection{Electrodes} \index{Electrodes}

\begin{center}
\includegraphics[width=0.45\textwidth]{./img/vso/electrodes.jpg}
\includegraphics[width=0.45\textwidth]{./img/vso/electrolyte-electrode.jpg}
\end{center}

\begin{description*}
%\item[Subtopic:]{}
%\item[Materials:]{}
%\item[Setup:]{}
%\item[Procedure:]{}
%\item[Hazards:]{}
%\item[Questions:]{}
%\item[Observations:]{}
\item[Theory:]{The materials shown here can be used as electrodes in electrolysis activities. Use the table to find the appropriate matching of electrolyte and electrodes.}
%\item[Applications:]{}
%\item[Notes:]{}
\end{description*}

%\subsection{Electrode Holders} % Put in sources of equip
%
%\begin{center}
%\includegraphics[width=0.4\textwidth]{./img/vso/electrode-holders.jpg}
%\end{center}
%
%\begin{description*}
%%\item[Subtopic:]{}
%\item[Materials:]{}
%\item[Setup:]{}
%\item[Procedure:]{}
%\item[Hazards:]{}
%\item[Questions:]{}
%\item[Observations:]{}
%\item[Theory:]{}
%\item[Applications:]{}
%\item[Notes:]{}
%\end{description*}

\subsection{Electrolysis Setups} \index{Electrolysis! setups of}

\begin{center}
\includegraphics[width=0.4\textwidth]{./img/vso/electrolysis.jpg}
\end{center}

\begin{description*}
%\item[Subtopic:]{}
%\item[Materials:]{}
%\item[Setup:]{}
\item[Procedure:]{Use any of the designs shown for setting up electrolysis experiments.}
%\item[Hazards:]{}
%\item[Questions:]{}
%\item[Observations:]{}
%\item[Theory:]{}
%\item[Applications:]{}
%\item[Notes:]{}
\end{description*}

%\subsection{Conservation of Energy} % already done in Fuels and Energy Form 2
%
%\begin{center}
%\includegraphics[width=0.45\textwidth]{./img/source/chem-energy.jpg}
%\end{center}
%
%\begin{description*}
%%\item[Subtopic:]{}
%\item[Materials:]{Copper (II) sulphate, zinc metal (from dry cell), copper wire, steel wool, ammeter/bulb}
%\item[Setup:]{Prepare a 2 M solution of copper (II) sulphate and clean pieces of copper wire and zinc using steel wool.}
%\item[Procedure:]{Connect the zinc anode, ammeter/bulb and copper cathode in series using connecting wires. Dip the zinc and copper electrodes into the copper (II) sulphate solution. Read the current on the ammeter.}
%%\item[Hazards:]{}
%%\item[Questions:]{}
%\item[Observations:]{The ammeter shows a deflection, possibly around 0.05 A.}
%\item[Theory:]{The current produced indicates that the chemical energy inherent in the electrodes and the electrolyte solution is converted to electrical energy.}
%%\item[Applications:]{}
%%\item[Notes:]{}
%\end{description*}

\vfill
\columnbreak

\subsection{Electrolytes in Food}

\begin{center}
\includegraphics[width=0.4\textwidth]{./img/lechlanche-cell.png}
\end{center}

\begin{description*}
%\item[Subtopic:]{}
\item[Materials:]{Lemons, zinc plate and carbon rod from old dry cell, connecting wires, galvanometer, bulb}
%\item[Setup:]{}
\item[Procedure:]{Make two holes in a lemon and insert the carbon rod and zinc plate into the holes. Connect the lemon to the galvanometer using connecting wires and notice any deflection that may occur. Repeat for several lemons by placing them in series and in parallel.}
%\item[Hazards:]{}
%\item[Questions:]{}
\item[Observations:]{The deflection increases with the number of lemons placed in series. With enough lemons, the bulb will light up.}
\item[Theory:]{Electric current can be produced from different cells - dry and wet. Wet cells can be made from natural foods such as lemons, Irish potatoes and salts which are strong electrolytes and hence produce electric current.}
%\item[Applications:]{}
%\item[Notes:]{}
\end{description*}

%==================================================================================================%

\section*{Mechanism of Electrolysis} \index{Electrolysis! mechanism of}


\subsection{Electrochemical Series} \index{Electrochemical series}

%\begin{center}
%\includegraphics[width=0.4\textwidth]{./img/.jpg}
%\end{center}

\begin{description*}
%\item[Subtopic:]{}
\item[Materials:]{Paper, marker pens, tape}
\item[Setup:]{Tape sheets labeled ``Cathode'' and ``Anode'' on opposite walls of the room. Make signs for students to represent cations and anions in a particular electrolyte. Each student should represent only one ion (e.g. Na$^,$ Cl$^-$, H$^+$ and OH$^-$ for NaCl solution). }
\item[Procedure:]{Students walk around the room. When you say go (representing the start of current flow), the students move to their respective electrodes (anions to anodes and cations to cathodes). Students (ions) lower in the electrochemical series must walk while those higher in the series run.}
%\item[Hazards:]{}
%\item[Questions:]{Which products are formed after the first ions arrive at each electrode?}
%\item[Observations:]{}
\item[Theory:]{Many factors affect ion discharge at electrodes, one of which is order of preference in the electrochemical series. Products at the electrodes depend on the discharged ions. Those at the top of the series take preference and are discharged instead of the lower ions.}
%\item[Applications:]{}
%\item[Notes:]{}
\end{description*}

\columnbreak

\subsection{Electrolysis of Water} \index{Water! electrolysis of}

\begin{center}
\includegraphics[width=0.45\textwidth]{./img/vso/electrolysis-nacl.jpg}
\end{center}

\begin{description*}
%\item[Subtopic:]{}
\item[Materials:]{Bottle, clothes pegs, dry cells, 2 syringes, speaker wire, bulb, water, table salt}
\item[Setup:]{Remove 2 carbon electrodes from old dry cell batteries. Connect the electrodes and bulb in series with 3-4 dry cells.}
\item[Procedure:]{Fix the electrodes in a cork or bottle top (with super glue to seal). Place an overturned empty syringe tube over each electrode. Fill the container with a dilute sodium chloride solution by dissolving table salt in water. Close the circuit.}
%\item[Hazards:]{}
%\item[Questions:]{}
\item[Observations:]{Bubbles at the electrodes indicate a reaction of electrolysis.}
\item[Theory:]{The cations present are H$^+$ from water and Na$^+$ from the sodium chloride. These migrate to the cathode, where the H$^+$ are discharged because hydrogen is lower than sodium in the reactivity series, and so \emph{hydrogen gas is formed at the cathode}. 

At the anode, OH$^-$ ions from water are discharged in favor of Cl$^-$ ions from salt, and so \emph{oxygen gas is formed at the anode}.

The complete chemical equation for this reaction is:\\

\ce{ 4H+_{(aq)} + 4OH^-_{(aq)} -> 2H2O_{(l)} + O2_{(g)} + 2H2_{(g)} }\\

The volume of hydrogen gas produced is twice as large as that of oxygen.}
%\item[Applications:]{}
%\item[Notes:]{}
\end{description*}

\vfill
\columnbreak

\subsection{Indicator Electrolysis} \index{Indicators! in electrolysis}

\begin{center}
\includegraphics[width=0.4\textwidth]{./img/electrolysis.jpg}
\end{center}

\begin{description*}
%\item[Subtopic:]{}
\item[Materials:]{Clear bottle, phenolphthalein (POP) indicator, water, salt, 4 dry cells (2 live, 2 dead) speaker wire}
\item[Setup:]{Strip the ends of the speaker wire. Carefully remove the carbon cores from the dead dry cells and wrap the stripped wires around the tops of them. In the container, make a saturated saline solution (keep dissolving salt until you can't dissolve anymore.) Add a few drops of POP indicator to the saline solution.}
\item[Procedure:]{Place the two carbon electrodes into the solution and attach them in series to the two remaining dry cells.}
\item[Hazards:]{Beware of battery acid from corroded batteries when removing the carbon cores. Additionally, in theory, the electrolytic cell will eventually produce dangerous chlorine gas (though this would require much time and a large setup).}
%\item[Questions:]{}
\item[Observations:]{The solution changes from transparent to a pink/purple colour as current continues to run through the solution. When the current is removed, the solution eventually returns to a colourless state.}
\item[Theory:]{As the current runs through the saline solution (electrolyte), the gases given off at the cathode and anode are hydrogen and oxygen respectively. While much of the hydrogen escapes as gas, some of the oxygen dissolves back in solution and causes it to become more basic (more OH$^-$ ions exist compared to H$^+$ ions). As the solution becomes more basic, it turns to purple due to the presence of PoP. (PoP turns pink/purple in the presence of bases.) Upon the removal of current, the dissolved oxygen escapes as oxygen gas or rejoins with hydrogen ions to reform water, causing the solution to return to a neutral pH.}
%\item[Applications:]{}
\item[Notes:]{The purple colour originates at the anode. After connecting and reconnecting the circuit multiple times, the solution will remain more basic and the colour will not return to being transparent.}
\end{description*}

\vfill
\columnbreak

%==================================================================================================%

%\section*{Laws of Electrolysis} % See LASM

%==================================================================================================%

\section*{Application of Electrolysis}


\subsection{Electroplating}

\begin{center}
\includegraphics[width=0.45\textwidth]{./img/vso/electroplating.jpg}
\end{center}

\begin{description*}
%\item[Subtopic:]{}
\item[Materials:]{Dry cells, iron nail, copper wire, speaker wire, copper (II) sulphate, water, bottle}
\item[Setup:]{Strip or scrape the insulation from the copper wire as shown and connect to the nail. Connect this end to the \emph{negative} terminal of the dry cell. To the positive terminal connect another stripped piece of copper wire. Place a copper (II) sulphate solution in the container. }
\item[Procedure:]{Submerge the nail and loose copper wire into the solution.}
%\item[Hazards:]{}
%\item[Questions:]{}
\item[Observations:]{In a short time, the nail becomes pinkish as copper deposits on its surface. If left for a long time, the loose copper wire will disappear.}
\item[Theory:]{The copper metal (anode) oxidizes to form Cu$^+$ ions, which migrate towards the cathode (iron nail) where reduction takes place. The copper ions gain electrons to once again form copper metal on the surface of the nail. }
\item[Applications:]{Chrome plating of jewelry, etc. uses this process with chromium in place of copper. Galvanized nails are iron nails with zinc electroplated onto their surface to prevent rusting.}
\item[Notes:]{Use any conducting object in place of the nail, e.g. spoon, graphite electrode, etc.}
\end{description*}


\end{multicols}

\pagebreak