\chapter{Sources of Chemicals}
\label{cha:sourcesofchemicals}
The following is a list of most of the chemicals 
used in science laboratories. 
Noted for each of the chemicals are local sources, 
low cost industrial sources, 
methods to manufacture them at school, 
and/or functional alternatives. 
Also listed is information such as other names, 
common uses 
and hazards. 
Finally, descriptions are included of many of the compounds 
and confirmatory tests to assist 
with identification of unlabeled chemicals. 
%For more information on this, 
%see \nameref{cha:unknownchemicals}.

Chemicals are generally listed alphabetically by IUPAC name, 
although many compounds are also cross-listed by their common name (e.g. 
acetone (common) / propanone (IUPAC)).\\

%The chemicals listed in this section are the following:\\
%\begin{multicols}{2}
%
%\begin{enumerate}
%\item \nameref{sec:methylpropanol}
%\item \nameref{sec:}
%\item \nameref{sec:}
%\item \nameref{sec:}
%\item \nameref{sec:}
%\item \nameref{sec:}
%\item \nameref{sec:}
%\item \nameref{sec:}
%\item \nameref{sec:}
%\item \nameref{sec:}
%\item \nameref{sec:}
%\item \nameref{sec:}
%\item \nameref{sec:}
%\item \nameref{sec:}
%\item \nameref{sec:}
%\item \nameref{sec:}
%\item \nameref{sec:}
%\item \nameref{sec:}
%\item \nameref{sec:}
%\item \nameref{sec:}
%\item \nameref{sec:}
%\item \nameref{sec:}
%\item \nameref{sec:}
%\item \nameref{sec:}
%\item \nameref{sec:}
%\item \nameref{sec:}
%\item \nameref{sec:}
%\item \nameref{sec:}
%\item \nameref{sec:}
%\item \nameref{sec:}
%\item \nameref{sec:}
%\item \nameref{sec:}
%\item \nameref{sec:}
%\item \nameref{sec:}
%\item \nameref{sec:}
%\item \nameref{sec:}
%\item \nameref{sec:}
%\item \nameref{sec:}
%\item \nameref{sec:}
%\item \nameref{sec:}
%\item \nameref{sec:}
%\item \nameref{sec:}
%\item \nameref{sec:}
%\item \nameref{sec:}
%\item \nameref{sec:}
%\item \nameref{sec:}
%\item \nameref{sec:}
%\item \nameref{sec:}
%\item \nameref{sec:}
%\item \nameref{sec:}
%\item \nameref{sec:}
%\item \nameref{sec:}
%\item \nameref{sec:}
%\item \nameref{sec:}
%\item \nameref{sec:}
%\item \nameref{sec:}
%\item \nameref{sec:}
%\item \nameref{sec:}
%\item \nameref{sec:}
%\item \nameref{sec:}
%\item \nameref{sec:}
%\item \nameref{sec:}
%\item \nameref{sec:}
%\item \nameref{sec:}
%\item \nameref{sec:}
%\item \nameref{sec:}
%\item \nameref{sec:}
%\item \nameref{sec:}
%\item \nameref{sec:}
%\item \nameref{sec:}
%\item \nameref{sec:}
%\item \nameref{sec:}
%\item \nameref{sec:}
%\item \nameref{sec:}
%\item \nameref{sec:}
%\item \nameref{sec:}
%\item \nameref{sec:}
%\item \nameref{sec:}
%\item \nameref{sec:}
%\item \nameref{sec:}
%\item \nameref{sec:}
%\item \nameref{sec:}
%\item \nameref{sec:}
%\item \nameref{sec:}
%\item \nameref{sec:}
%\item \nameref{sec:}
%\item \nameref{sec:}
%\item \nameref{sec:}
%\item \nameref{sec:}
%\item \nameref{sec:}
%\item \nameref{sec:}
%\item \nameref{sec:}
%\item \nameref{sec:}
%\item \nameref{sec:}
%\item \nameref{sec:}
%\item \nameref{sec:}
%\item \nameref{sec:}
%\item \nameref{sec:}
%\item \nameref{sec:}
%\item \nameref{sec:}
%\item \nameref{sec:}
%\item \nameref{sec:}
%\item \nameref{sec:}
%\item \nameref{sec:}
%\item \nameref{sec:}
%\item \nameref{sec:}
%\item \nameref{sec:}
%\item \nameref{sec:}
%\item \nameref{sec:}
%\item \nameref{sec:}
%\item \nameref{sec:}
%\item \nameref{sec:}
%\item \nameref{sec:}
%\item \nameref{sec:}
%\item \nameref{sec:}
%\end{enumerate}
%
%\end{multicols}

\begin{multicols}{2}

\section{2-methylpropanol}
\label{sec:methylpropanol}
Formula: \ce{(CH3)2CHCH2OH}\\
Other names: isobutanol\\
Description: clear liquid less dense than water, 
alcohol smell similar to isopropanol (American rubbing alcohol)\\
Use: organic solvent for distribution (partition) experiments\\
Alternative: paint thinner or kerosene\\
Note: if ordering this chemical for the national exam, 
make sure that you get this chemical exactly. 
Other compounds, 
e.g. 
\ce{CH3CH2CH2(OH)CH3} (butan-2-ol) are sometimes sold 
as ‘isobutanol’ but do not work the same way.

\section{Acetaldehyde}
%\label{sec:}
See \nameref{sec:ethanal}.

\section{Acetic acid}
%\label{sec:}
See \nameref{sec:ethanoic}.

\section{Acetone}
%\label{sec:}
See \nameref{sec:propanone}.

\section{Alum}
%\label{sec:}
See \nameref{sec:potalsulf}.

\section{Ammonia solution}
\label{sec:ammoniasol}
Formula: \ce{NH3_{(aq)}}\\
Other names: ammonium hydroxide, 
ammonium hydroxide solution\\
Description: clear liquid less dense than water, 
completely miscible in water, 
strong biting smell similar to old urine\\
Use: qualitative analysis, various experiments\\
Source: released from an aqueous mixture of ammonium salt and hydroxide, 
for example calcium ammonium nitrate and sodium hydroxide. 
The gas can be trapped and dissolved in water.\\
Alternative: to distinguish between zinc and lead cations, 
add dilute sulfuric acid dropwise. 
The formation of a white precipitate -- lead sulfate -- confirms lead.
Note: ammonia solution also is called ammonium hydroxide 
because ammonia undergoes autoionization to form ammonium and hydroxide ions. 
Just like water, 
there is an equilibrium concentration of the ions in an ammonia solution.

\section{Ammonium dichromate}
%\label{sec:}
Formula: \ce{(NH3)2Cr2O7}\\
Description: orange crystals soluble in water\\
Use: qualitative analysis (identification of sulfur dioxide gas)\\
Hazard: toxic, 
water pollutant\\
Alternative: make ammonium/potassium dichromate paper tests. 
Many can be made from a single gram of ammonium/potassium dichromate.

\section{Ammonium hydroxide solution}
%\label{sec:}
See \nameref{sec:ammoniasol}.

\section{Ammonium carbonate, chloride, and nitrate}
%\label{sec:}
Use: qualitative analysis, 
preparation of ammonia\\
Alternative: to teach the identification 
and confirmation of ammonium salts and to prepare ammonia, 
use calcium ammonium nitrate.

\section{Ammonium sulphate}
%\label{sec:}
Formula: \ce{(NH4)2SO4}
Other name: sulphate of ammonia
Description: white crystals
Use: qualitative analysis, preparation of ammonia\\
Source: fertilizer

\section{Ammonium thiocyanate}
%\label{sec:}
Formula: \ce{NH4SCN}\\
Use: confirmation of iron III in qualitative analysis\\
Alternative: addition of sodium ethanoate 
should also produce a blood red solution; 
additionally, 
the test is unnecessary, 
as iron III is also the only chemical 
that will produce a red/brown precipitate with sodium hydroxide solution 
or sodium carbonate solution.

\section{Ascorbic acid}
%\label{sec:}
Other names: vitamin C\\
Formula: \ce{C6H7O7}\\
Description: white powder, 
but pharmacy tablets often colored\\
Confirm: aqueous solution turns blue litmus red 
AND decolorizes dilute iodine or potassium permanganate solution\\
Use: all-purpose reducing agent, 
may substitute for sodium thiosulfate in redox titrations, 
removes iodine and permanganate stains from clothing\\
Source: pharmacies

\section{Barium chloride and barium nitrate}
%\label{sec:}
Use: confirmatory test for sulfate in qualitative analysis\\
Description: white crystals\\
Hazard: toxic, 
water pollutant\\
Alternative: lead nitrate will precipitate lead sulfate -- 
results identical to when using barium

\section{Boric acid}
%\label{sec:}
Formula: \ce{H3BO3}\\
Description: white powder\\
Confirm: deep green flame color\\
Use: flame test demonstrations, preparation of sodium borate\\
Source: village industry supply shops, industrial chemical

\section{Benedict's solution}
\label{sec:benedict}
Description: bright blue solution\\
Confirm: gives orange precipitate when boiled with glucose\\
Use: food tests (test for reducing and non reducing sugars)\\
Hazard: copper is poisonous\\
Manufacture: combine 5 spoons of sodium carbonate, 
3 spoons of citric acid, 
and one spoon of copper sulfate in half a liter of water. 
Shake until everything is fully dissolved.

\section{Benzene}
%\label{sec:}
Formula: \ce{C6H6}\\
Description: colorless liquid insoluble in water\\
Use: all purpose organic solvent\\
Hazard: toxic, 
highly carcinogenic -- see section on Dangerous Chemicals\\
Alternative: toluene is safer but for most solvent applications 
kerosene is equally effective and far less expensive.

\section{Butane}
%\label{sec:}
Formula: \ce{C4H10}\\
Source: the fluid in gas lighters is butane under pressure; 
liquid butane may be obtained at normal pressure with the help of a freezer

\section{Calcium ammonium nitrate}
%\label{sec:}
Other names: \ce{CAN}\\
Description: small pellets, 
often with brown coating; 
endothermic heat of solvation\\
Use: low cost ammonium salt for teaching qualitative analysis; 
not as useful for teaching about nitrates 
as no red/brown gas released when heated. 
May be used for the preparation of ammonia and sodium nitrate.\\
Source: agricultural shops (fertilizer)\\

\section{Calcium carbonate}
%\label{sec:}
Formula: \ce{CaCO3}\\
Description: white powder, 
insoluble in water
Confirm: brick red flame test and acid causes effervescence\\
Use: demonstration of reactivity of carbonates, 
rates of reaction, 
qualitative analysis\\
Source: coral rock, 
sea shells, 
egg shells, 
limestone, 
marble, 
white residue from boiling water\\
Local manufacture: prepare a solution of aqueous calcium 
from either calcium ammonium nitrate or calcium hydroxide 
and add a solution of sodium carbonate.\\ 
Calcium carbonate will precipitate and may be filtered and dried.

\section{Calcium chloride and calcium nitrate}
%\label{sec:}
Description: highly deliquescent colorless crystals 
(poorly sealed containers often become thick liquid)\\
Use: qualitative analysis salts, 
drying agents\\
Alternatives (qualitative analysis): 
to practice identification of the calcium cation, 
use calcium sulfate; 
to practice identification of the chloride anion, 
use sodium chloride\\
Alternative (drying agent): sodium sulfate

\section{Calcium hydroxide}
%\label{sec:}
Formula: \ce{Ca(OH)2}\\
Other names: quicklime\\
Local name: \textit{chokaa}\\
Description: white to off white powder, 
sparingly soluble in water\\
Use: dissolve in carbonate-free water to make limewater\\
Source: building supply shops\\
Alternative: add a small amount of cement to water, 
let settle, 
and decant the clear solution; 
this is limewater.

\section{Calcium oxide}
%\label{sec:}
Formula: \ce{CaO}\\
Other names: lime\\
Use: reacts with water to form calcium hydroxide, 
thus forming limewater\\
Source: cement is mostly calcium oxide

\section{Calcium sulfate}
%\label{sec:}
Formula: \ce{CaSO4.2H2O}\\
Other names: gypsum, 
plaster of Paris\\
Description: white powder, 
insoluble in cold water but soluble in hot water\\
Use: qualitative analysis\\
Source: building supply companies (as gypsum powder)

\section{Carbon (amorphous)}
%\label{sec:}
Source: soot, 
charcoal (impure)

\section{Carbon (graphite)}
\label{sec:carbongraphite}
Use: element, \\
inert electrodes for chemistry and physics
Source: dry cell battery electrodes, 
pencil cores (impure)

\section{Carbon dioxide}
%\label{sec:}
Preparation: react an aqueous weak acid 
(citric acid or ethanoic acid) with a soluble carbonate 
(sodium carbonate or sodium hydrogen carbonate)

\section{Carbon tetrachloride}
%\label{sec:}
See \nameref{sec:tetrachloromethane}.

\section{Chloroform}
%\label{sec:}
See \nameref{sec:trichloromethane}.

\section{Citric acid}
%\label{sec:}
Formula: \ce{C6H8O7} = \ce{CH2(COOH)COH(CHOOH)CH2COOH}\\
Local name: \textit{unga wa ndimu}\\
Description: white crystals soluble in water, 
endothermic heat of solvation\\
Use: all purpose weak acid, 
volumetric analysis, 
melting demonstration, 
preparation of carbon dioxide, 
manufacture of Benedict's solution\\
Hazard: acid – keep out of eyes!\\
Source: markets (sold as a spice), 
supermarkets

\section{Cobalt chloride}
%\label{sec:}
Use: test for water (hydrated cobalt chloride is pink)\\
Hazard: cobalt is poisonous\\
Alternative: white anhydrous copper sulfate turns blue when hydrated

\section{Copper}
\label{sec:copper}
Use: element, 
preparation of copper sulfate, 
electrochemical reactions\\
Description: dull red/orange metal\\
Source: electrical wire -- e.g. 
2.5~mm gray insulated wire has 50~g of high purity copper per meter.\\
Note: modern earthing rods are only copper plated, 
and thus no longer a good source of copper

\section{Copper carbonate}
%\label{sec:}
Formula: \ce{CuCO3}\\
Description: light blue powder\\
Confirm: blue/green flame test and dilute acid causes effervescence\\
Use: qualitative analysis, 
preparation is a demonstration of double decomposition\\
Hazard: powder may be inhaled; 
copper is poisonous\\
Local manufacture: prepare solutions of copper sulfate 
and sodium carbonate and mix them. 
Copper carbonate will precipitate 
and may be purified by filtration and drying.

\section{Copper chloride and copper nitrate}
%\label{sec:}
Description: blue-green (copper chloride) 
and deep blue (copper nitrate) salts \\
Use: qualitative analysis\\
Alternatives: for practice identifying the copper cation, 
use copper sulfate; 
for practice identifying the chloride anion, 
use sodium chloride

\section{Copper oxygen chloride}
%\label{sec:}
Formula: \ce{Cu2OCl}\\
Other names: copper oxychloride, 
blue copper\\
Description: light blue powder\\
Hazard: powder may be inhaled; 
copper is poisonous\\
Source: agricultural shops (fungicide)

\section{Copper sulfate}
%\label{sec:}
Formula: \ce{CuSO4} (anhydrous), 
\ce{CuSO4.5H2O} (pentahydrate)\\
Local name: \textit{mlutulutu}\\
Description: white (anhydrous) or blue (pentahydrate) crystals\\
Confirm: blue/green flame test 
and aqueous solution gives a white precipitate 
when mixed with lead or barium solution\\
Use: qualitative analysis, 
demonstration of the reactivity series, 
manufacture of Benedict's solution, 
test for water\\
Source: imported ``local'' medicine (manufactured in India).\\ 
Local manufacture: Electrolyze dilute (1-2~M) sulfuric acid 
with a copper anode and inert (e.g. 
graphite) cathode. 
Evaporate final solution until 
blue crystals of copper sulfate pentahydrate precipitate. 
To prepare anhydrous copper sulfate from copper sulfate pentahydrate, 
gently heat until the blue color has faded. 
Strong heating will irreversibly form black copper oxide. 
Store anhydrous copper sulfate in an air-tight container -- 
otherwise atmospheric moisture will reform the pentahydrate.

\section{Dichloromethane}
%\label{sec:}
Formula: \ce{CH2Cl2}\\
Use: organic solvent for distribution (partition) experiments\\
Hazard: toxic by inhalation and ingestion (mouth pipetting) 
and by absorption though skin\\
Alternative: paint thinner or kerosene, 
although these are less dense than water

\section{Diethyl ether}
%\label{sec:}
Formula: \ce{(CH3CH2)2O}\\
Description: colorless liquid with smell similar to nail polish remover, 
evaporates quickly at room temperature\\
Use: organic solvent for distribution (partition) experiments, 
demonstration of low boiling point\\
Hazard: extremely flammable (boils near room temperature) 
and dangerous to inhale (unfortunate as it is very volatile!). 
It is of the utmost importance not to mouth pipette this chemical. 
Breathing ether was the first anesthesia, 
discontinued because it can be lethal.\\
Alternatives (distribution/partition): paint thinner or kerosene\\
Alternative (low boiling point): propanone

\section{Distilled water}
%\label{sec:}
Formula: \ce{H2O} and nothing else!\\
Local name: \textit{maji baridi}\\
Use: qualitative analysis\\
Source: rain water.\\
Allow the first 15 minutes of rain to clean off the roof 
and then start collecting water. 
In schools in dry climates, 
collect as much rain water as possible during the rainy season. 
Use it only for qualitative analysis, 
preparation of qualitative analysis reagents, 
and manufacture of qualitative analysis salts.\\ 
Distilled water may also be purchased at most petrol stations 
and automotive shops.\\
Local manufacture: Heat water in a kettle 
and use a rubber hose to bring the steam through a container of cold water. 
Collect the condensate -- pure water.\\
Alternative: river or tap water is almost always sufficient. 
Volumetric analysis never needs distilled water 
if you follow the instructions in Relative Standardization. 
Also, 
the tap water in many places is sufficient for even qualitative analysis.

\section{Ethanal}
\label{sec:ethanal}
Formula: \ce{CH3CHO}\\
Other names: acetaldehyde\\
Description: clear liquid with a foul smell\\
Local manufacture: oxidize ethanol with potassium permanganate\\
Note: the product is truly bad smelling and probably unhealthy to inhale. 
Include this entry only to show that rather than useful ethanoic acid, 
one can only get useless ethanal by chemical oxidation of ethanol; 
manufacture of ethanoic acid requires elevated temperature 
and high pressure vessels (or biology, 
as in the traditional manufacture of vinegar). 
The reaction at small scale 
(1~mL of ethanol used to decolorize dilute potassium permanganate) 
is useful when teaching oxidation of alcohols in organic chemistry.

\section{Ethandioic acid}
%\label{sec:}
Formula: \ce{C2H2O4.2H2O}\\
Other names: oxalic acid\\
Description: clear crystals\\
Use: volumetric analysis, 
primary standard for absolute standardization, 
reducing agent (oxidized to carbon dioxide)\\
Hazard: poisonous (also acidic)\\
Alternative: substitute citric acid or ethanoic acid 
for weak acid solutions and use ascorbic acid as a reducing agent.

\section{Ethanoic acid}
\label{sec:ethanoic}
Formula: \ce{CH3COOH}\\
Other names: acetic acid\\
Description: clear liquid, 
completely miscible with water, 
strong vinegar smell\\
Use: all purpose weak acid, 
volumetric analysis\\
Source: 96\% solution available from village industry supply shops, 
vinegar (5\% solution) available in small shops and supermarkets\\
Safety for 96\% ethanoic acid: HARMFUL VAPORS. 
Use outside or in a well ventilated space. 
CORROSIVE ACID. 
Always have dilute weak base solution (e.g. 
sodium hydrogen carbonate) available to neutralize spills. 
Wear gloves and goggles when handling. 
Do not induce vomiting if ingested.\\
Alternative: for a weak acid, 
citric acid. 

\section{Ethanol}
%\label{sec:}
Formula: \ce{CH3CH2OH}\\
Description: clear liquid, 
completely miscible with water, 
strong and sweet alcohol smell\\
Use: solvent, 
extraction of chlorophyll, 
removes permanent marker, 
preparation of POP solution, 
distillation, 
preservation of biological specimens\\
Hazard: ethanol itself is a mild poison, 
and methylated spirits and other industrial alcohol contain 
additional poisonous impurities (methanol) 
specifically so that no one drinks it\\
Sources: methylated spirits are 70\% ethanol, 
hard liquor is often 30-40\%, 
village-brewed concentrated alcohol varies 
and may contain toxic quantities of methanol\\
Local manufacture: fermentation of sugar by yeast will produce 
up to a 15\% solution -- at that point, 
the yeast dies; 
distillation can in theory concentrate this to up to 95\%, 
but this is hard with simple materials. 
Nevertheless, 
preparing ethanol of sufficient concentration to dissolve POP (50-60\%) 
is quite possible.\\
Note: the color of most methylated spirits makes them undesirable 
for preparation of POP; 
hard liquor will suffice, 
but poorly because of its relatively low ethanol content. 
Colored methylated spirits can be run 
through a simple distillation apparatus to produce colorless spirits, 
as the pigment is less volatile than the ethanol. 
Of course, 
methanol and other poisons remain, 
but the clear solution works beautifully for dissolving POP.\\ 
Beware that ethanol vapors are flammable -- 
a poorly constructed distillation setup may explode.

\section{Ethyl acetate}
%\label{sec:}
See \nameref{sec:ethylethanoate}.

\section{Ethyl ethanoate}
\label{sec:ethylethanoate}
Formula: \ce{CH3COOCH2CH3}\\
Other names: ethyl acetate\\
Description: clear liquid, 
immiscible with water, 
smells like nail polish remover\\
Use: solvent\\
Source: nail polish remover (mixture with propanone)\\
Alternative: paint remover, 
paint thinner, 
or methylated spirits\\
Preparation (demonstration of esterification): 
mix ethanol and ethanoic acid 
with a catalytic amount of strong acid or base; 
the decrease in ethanoic acid can be detected 
by titration and the ethyl ethanoate can be detected by smell.

\section{Gelatin}
%\label{sec:}
Source: may be extracted from chicken bones. 
This process is lengthy compared 
to purchasing gelatin powder from supermarkets. 
Be sure to purchase the non flavored varieties, 
usually in white boxes.

\section{Glucose}
%\label{sec:}
Formula: \ce{C6H12O6}\\
Description: white powder\\
Use: food tests (biology), 
reducing agent\\
Sources: small shops, 
pharmacies\\
Note: for food tests, 
the vitamins added to most glucose products will not cause a problem

\section{Gold}
%\label{sec:}
Source: a very thin coat of gold is plated 
onto the electrical contacts of cell phone batteries 
and mobile phone SIM cards.

\section{Graphite}
%\label{sec:}
See \nameref{sec:carbongraphite}.

\section{Hydrochloric acid}
\label{sec:hydroacid}
Formula: \ce{HCl}, 
36.5~g/mol, 
density 1.18~g/cm$^{3}$ when concentrated ($\sim$12~M)\\
Other names: muriatic acid, 
pH decreasing compound for swimming pools\\
Description: clear liquid, 
may be discolored by contamination, 
distinct smell similar to chlorine 
although sometimes smells strongly of vinegar\\
Confirm: decolorizes weak solutions of potassium permanganate; 
white precipitate in silver nitrate solution 
and effervescence with (hydrogen) carbonates\\
Use: volumetric analysis, 
qualitative analysis\\
Source: swimming pool chemical suppliers (impure), industrial chemical (concentrated)\\ 
Safety: HARMFUL VAPORS. 
Use outside or in a well ventilated space. 
CORROSIVE ACID. 
Always have dilute weak base solution (e.g. 
sodium hydrogen carbonate) available to neutralize spills. 
Wear gloves and goggles when handling. 
Extremely toxic hydrogen cyanide gas formed 
on mixing with cyanides or hexacyanoferrate compounds. 
Toxic chlorine gas formed on reaction with oxidizing agents, 
especially bleach. 
Do not induce vomiting if ingested.\\
Alternative (strong acid): sulfuric acid\\
Alternative (acid): citric acid\\
Alternative (qualitative analysis): for the test for carbonates, 
use dilute sulfuric acid; 
to dissolve insoluble carbonates, 
nitric acid may be used instead

\section{Hydrogen}
%\label{sec:}
Formula: \ce{H2}\\
Confirm: ``pop sound,'' i.e. 
ignites with a bang; 
in an inverted test tube the rapid movement of air 
near the mouth creates a rapid, 
high pitch ``whoosh'' that gives the ``pop'' name\\
Preparation: combine dilute acid (e.g. 
battery acid) and a reactive metal (steel wool or zinc) 
in a plastic water bottle. 
Attach a balloon to the top of the water bottle; 
being less dense than air, 
hydrogen will migrate up and slowly fill the balloon. 
Specific instructions for various alternatives are available 
in the Hands-On activities section. 
Before ignition, 
always move the balloon away from the container of acid.

\section{Hydrogen peroxide}
%\label{sec:}
Formula: \ce{H2O2}\\
Local name: \textit{dawa ya vidonda}\\
Description: solutions are colorless liquids 
appearing exactly like water\\
Confirm: decolorizes potassium manganate (VII) solution 
in the absence of acid, 
neutral pH\\
Use: preparation of oxygen, 
general oxidizer and also may act as a reducing agent (e.g. 
with potassium permanganate)\\
Source: pharmacies sell 3\% (10 volume) and 6\% (20 volume) solutions 
as medicine for cleaning sores\\
Note: `20 volume' means it will produce 20 times its liquid volume in oxygen gas.

\section{Hydrogen sulfide}
%\label{sec:}
Formula: \ce{H2S}\\
Description: colorless gas with the smell of rotting eggs, 
ocean mud, 
and other places of anaerobic respiration\\
Safety: the gas is quite poisonous, 
although the body can detect extremely small amounts\\
Preparation: a sufficient quantity to smell 
may be prepared by igniting sulfur in a spoon 
and then quenching it in water.

\section{Indicator}
\label{sec:indicator}
Source: red flowers\\
Preparation: Crush flower petals in water. 
Some effective flowers include rosella, 
bougainvillea, 
and hibiscus. 
Test other flowers near your school.\\
Note: For bougainvillea and some other flowers, 
extract the pigment with ethanol 
or hard alcohol to get a better color. 
Color will change from pink (acidic) to colorless (basic). 
Rosella will change from red (acidic) to green (basic).
For an indicator in redox titrations involving iodine, 
see starch solution.

\section{Iodine}
%\label{sec:}
Formula: \ce{I2_{(s)}}\\
Description: purple/black crystals\\
Local manufacture: add a little dilute sulfuric acid 
to iodine solution from a pharmacy. 
Then add sodium hypochlorite solution (bleach) dropwise 
until the solution turns colorless with solid iodine resting on the bottom. 
The solid iodine can be removed by filtration or decantation. 
If pure iodine is necessary, 
the solid may be purified by sublimation.\\
Note: this reaction produces poisonous chlorine gas. 
Therefore, 
produce iodine in a well ventilated area and stand upwind.

\section{Iodine solution}
\label{sec:iodinesol}
Composition: \ce{I2} + \ce{KI} dissolved in water and sometimes ethanol\\
Description: light brown solution\\
Confirm: turns starch blue or black\\
Use: food tests for detection of starch and fats\\
Source: pharmacies sell a ‘weak iodine solution’ 
or ‘tincture of iodine’ that is really about 50\% by mass iodine. 
To prepare a useful solution for food tests, 
dilute this 10:1 in ordinary water.\\
Note: to use this solution for detection of fats, 
it must be made without ethanol, 
spirits, 
alcohol and the like. 
Either kind works for detection of starch.

\section{Iron}
\label{sec:iron}
Use: element, 
demonstration of reactivity series, 
preparation of hydrogen, 
preparation of iron sulfide, 
preparation of iron sulfate\\
Source: for samples of the element 
and for use in electrochemical experiments, 
buy non-galvanized nails at a hardware store, 
or find them on the ground. 
You can tell they are not galvanized because they are starting to rust. 
Clean off the rust with steel wool prior to use. 
For samples of the element for preparation of other compounds, 
buy steel wool from small shops or supermarkets. 
This has a very high surface area / mass ratio, 
allowing for faster reactions.

\section{Iron sulfate}
%\label{sec:}
Description: iron (II) sulfate is light green. 
If exposed to air and especially water, 
iron (II) sulfate oxidizes to form yellow/red/brown iron (III) sulfate.\\
Use: oxidation-reduction experiments, 
qualitative analysis\\
Local manufacture: add excess steel wool to battery acid 
and leave overnight or until the acid is completely consumed. 
Beware! This reaction produces poisonous sulfur dioxide gas! 
Decant the solution of iron sulfate and leave to evaporate. 
Gentle heating is useful to speed up evaporation, 
but be careful to not heat too strongly once crystals form.\\
Note: the product may contain both iron II sulfate and iron III sulfate -- 
you can guess based on the color. 
Such a mixture may be used to demonstrate confirmation of iron 
with potassium hexacyanoferrate (II/III), 
though not the specificity of one versus the other. 
To see if any iron II sulfate is present, 
add a solution of the product 
to a very dilute solution of potassium permanganate. 
If the permanganate is decolorized, 
iron (II) is present. 
If the solid has any yellow or red color, 
iron (III) is present.

\section{Iron sulfide}
%\label{sec:}
Use: preparation is a demonstration of chemical changes\\
Preparation: grind steel wool into a fine powder 
and mix with a similar quantity of sulfur. 
This is a mixture that may be physically sorted (e.g. 
with a magnet). 
Now, 
heat the mixture in a spoon over a flame. 
Iron sulfide will form. 
This is a chemical compound; 
the iron and sulfur can no longer be separated by physical means.

\section{Isobutanol}
%\label{sec:}
See \nameref{sec:methylpropanol}.

\section{Lead}
%\label{sec:}
Description: soft, 
dull gray metal\\
Hazard: toxic, 
especially its soluble compounds (e.g. 
lead acetate, 
chloride, 
and nitrate) and in powder form (e.g. 
lead carbonate)\\
Use: element\\
Source: electrodes from old car batteries; 
the old batteries themselves may be purchases from scrap dealers. 
Remember that the electrolyte may still be 5~M sulfuric acid 
and thus great care is required 
when opening these batteries to extract the electrodes. 
If you pay someone else to extract them, 
make sure they understand the hazards and use protective gear (gloves, 
goggles, 
etc.).

\section{Lead nitrate}
%\label{sec:}
Formula: \ce{Pb(NO3)2}\\
Use: qualitative analysis salt, 
alternative to barium chloride/nitrate when confirming sulfates\\
Hazard: toxic, 
water pollutant\\
Note: Yes, 
you could prepare this from lead metal and dilute nitric acid, 
and yes, 
this would be less expensive than buying lead nitrate. 
However, 
the process of dissolving a reactive metal in a highly corrosive acid 
to produce a toxic salt is anything but safe. 
Lead nitrate is a good chemical to purchase. 
Note that lead does not react with concentrated nitric acid.

\section{Lead shot}
%\label{sec:}
Use: very dense material for building hydrometers, 
etc.\\
Source: shotgun shells from a firearm shop -- 
ask them to open them for you\\
Note: most lead shot these days is actually a bismuth compound 
to reduce the environmental pollution of spraying lead everywhere. 
To test the lead shot, 
put in a ceramic or metal container 
and heat over a charcoal or kerosene stove. 
If the metal is lead, 
it will melt. 
Bismuth melts at a much higher temperature.\\
Alternative: If you just need a dense material for physics experiments, 
use iron and adjust the calibration. 
This is both safer and less expensive. 
If you need lead as a chemical reagent 
1) see the entry for lead but 
2) consider another demonstration with a less poisonous material.

\section{Lithium ions}
%\label{sec:}
Use: flame test demonstrations\\
Source: broken cell phone batteries from a phone repair shop\\
Extraction: Open the metal battery case by chipping 
or smashing it and then prying it open with pliers. 
There should be sealed packets inside. 
Stand upwind and cut these open; 
leave the contents to evaporate the noxious solvent for a few minutes. 
Do not breathe the fumes. 
After waiting ten minutes, 
remove the contents of the packets with pliers 
and unroll a strip of black covered silvery metal foil. 
Somewhere in here is some lithium ion. 
We used to think the silvery metal was lithium. 
That seems to be incorrect. 
Regardless, put some of the metal and the black coating
into a really hot flame (Bunsen burner, gas lighter)
and you should get the crimson flame color
characteristic of lithium.

\section{Magnesium carbonate}
%\label{sec:}
Use: preparation is a demonstration of double displacement reactions 
as well as a qualitative analysis test\\
Local manufacture: Mix a solution of magnesium sulfate 
with a solution of sodium carbonate. 
Manganese carbonate will precipitate and may be filtered and dried.

\section{Magnesium sulfate}
\label{sec:magsulfate}
Formula: \ce{MgSO4.7H2O}\\
Other names: epsom salts\\
Description: white or clear crystals\\
Use: crystallization experiments, 
qualitative analysis test reagent 
(confirmation of hydrogen carbonate and carbonate), 
precipitation reactions\\
Source: livestock and veterinary supply shops sell Epsom salts 
to treat constipation in cattle

\section{Manganese (IV) oxide}
%\label{sec:}
Formula: \ce{MnO2}\\
Other names: manganese dioxide\\
Description: black powder\\
Confirm: liberates oxygen from hydrogen peroxide\\
Use: preparation of oxygen, 
qualitative analysis (confirmation of chlorides)\\
Source: old dry cell batteries (radio batteries)\\
Extraction: smash a dry cell battery with a rock 
and scrape out the black powder. 
This is a mixture of manganese dioxide, 
zinc chloride, 
and ammonium chloride. 
This impure mixture is suitable for the preparation of oxygen. 
To purify manganese dioxide for use in qualitative analysis, 
boil the powder in water to dissolve away the chlorides. 
Filter the solution after boiling 
and repeat if the test gives false positives (e.g. 
confirms chlorides in samples that lack chlorides)\\
Note: Wash your hands with soap if you accidentally touch the powder. 
Do not get it on your clothes or into cuts on your hands. 
\ce{MnO2} causes metal to corrode; 
if you use a metal tool to scrap out the powder, 
be sure to clean it off afterwards. 
Better: use non-metal tools. 

\section{Methane}
%\label{sec:}
Formula: \ce{CH4}\\
Other names: natural gas\\
Use: optimal Bunsen burner fuel\\
Local manufacture: biogas systems --
a school could in theory build one of these to supply gas for Bunsen burners\\
Alternative: compressed gas, 
propane, 
may be purchased in most towns; 
this is generally how schools operate Bunsen burners

\section{Millon's reagent}
%\label{sec:}
Composition: mercury metal dissolved in nitric acid\\
Description: clear liquid, 
very low pH, 
addition of excess sodium hydroxide to a small sample 
produces a yellow precipitate (of toxic mercury hydroxide)\\
Use: identification of proteins in food tests\\
Hazard: highly toxic and very corrosive -- never use\\
Alternative: sodium hydroxide solution 
and copper sulfate solution in the Biuret test 
(1~M \ce{NaOH} followed by 1\% \ce{CuSO4})

\section{Naphthalene}
%\label{sec:}
Formula: \ce{C10H8}\\
Description: solid at room temperature but melts in boiling water, 
distinct smell of moth balls\\
Use: melting point and heat of fusion experiments\\
Source: moth balls are just solid naphthalene\\
Hazard: poison, 
possible carcinogen\\
Alternative: vaseline from small shops is 
another solid at room temperature that melts in boiling water

\section{Nestler's reagent}
%\label{sec:}
Description: colorless liquid, 
sometimes with a precipitate at the bottom; 
addition of excess sodium hydroxide to a small sample 
produces a yellow precipitate (toxic mercury hydroxide)\\
Use: detection of ammonia\\
Hazard: contains dissolved mercury -- very toxic\\
Alternative: ammonia is readily detected by smell; 
a possible ammonia solution can be confirmed by adding it drop-wise 
to a solution of copper sulfate -- 
a blue precipitate should form 
which then dissolves in excess ammonia to form a deep blue / purple solution.

\section{Nitric acid}
%\label{sec:}
Formula: \ce{HNO3}\\
Description: clear liquid though may turn yellow over time, 
especially if left in the light\\
Use: various experiments, 
qualitative analysis, 
cleaning stubborn residues\\
Hazard: highly corrosive acid; 
dissolves essentially everything in the laboratory except glass, 
ceramics, 
and many kinds of plastic; 
may convert organic material into explosives\\
Alternative (strong acid): battery acid\\
Alternative (qualitative analysis): 
have students practice dealing with insoluble carbonates by using copper, 
iron, 
or zinc carbonates that will dissolve in dilute sulfuric acid\\
Alternative (cleaning glassware): 
make residues in metal spoons that can be cleaned easily by abrasion

\section{Organic solvents}
%\label{sec:}
Sources: kerosene, 
petrol, 
paint remover, 
paint thinner and the safest: cooking oil

\section{Oxygen}
%\label{sec:}
Confirm: oxygen gas relights a glowing splint, 
i.e. 
a piece of wood or paper glowing red / orange 
will flame when put in a container 
containing much more oxygen than the typical 20\% in air\\
Preparation: combine hydrogen peroxide 
and manganese (IV) oxide in a plastic water bottle. 
Immediately crush the bottle to remove all other air and then cap the top. 
The bottle will re-inflate with oxygen gas.

\section{Phosphorus}
%\label{sec:}
Use: element\\
Source: the strike pads for matches contain impure red phosphorus

\section{Potassium aluminum sulfate}
\label{sec:potalsulf}
Formula: \ce{KFe(SO4)2}\\
Other names: potassium alum\\
Local name: \textit{shaabu}\\
Description: colorless to white crystals, 
sometimes very large, 
quite soluble in water\\
Use: coagulant useful in water treatment -- 
a small amount will precipitate all of the dirt in a bucket of dirty water\\
Source: various shops, 
especially those specializing in tradition ``Arab'' of ``Indian'' products

\section{Potassium carbonate}
%\label{sec:}
Formula: \ce{K2CO3}\\
Other names: potash\\
Description: white powder\\
Use: volumetric analysis\\
Safety: rather caustic, keep off of hands and definitely out of eyes!\\
Alternative: sodium carbonate -- 
see \nameref{cha:subchemvolana}.

\section{Potassium chromate}
%\label{sec:}
Formula: \ce{K2CrO4}\\
Description: yellow crystals soluble in water\\
Hazard: poison, 
water pollutant\\
Use: demonstration of reversible reactions, 
qualitative analysis (confirmation of lead)\\
Alternative (reversible reactions):
Dehydrate hydrated copper (II) sulfate by heating 
and then rehydrate it by adding drops of water\\
Alternative (confirmation of lead): 
Confirm lead by the addition of dilute sulfuric acid -- 
white lead sulfate precipitates

\section{Potassium dichromate}
%\label{sec:}
Formula: \ce{K2Cr2O7}\\
Description: orange crystals soluble in water\\
Use: demonstration of chemical equilibrium, 
qualitative analysis (identification of sulfur dioxide gas)\\
Hazard: toxic, 
water pollutant\\
Alternative: make ammonium / potassium dichromate paper tests. 
Many can be made from a single gram of ammonium/potassium dichromate.

\section{Potassium hexacyanoferrate (II)}
%\label{sec:}
Formula: \ce{K4Fe(CN)6}\\
Other name: potassium ferrocyanide\\
Description: pale yellow salt\\
Use: confirmatory tests in qualitative analysis 
(forms an intensely blue precipitate with iron (III) ions, 
a red-brown precipitate with copper, 
and a blue-white precipitate with zinc\\
Alternative (confirmation of iron (III) ions): 
see possibilities listed with ammonium thiocyanate\\
Alternative (confirmation of copper): blue/green flame test, 
blue precipitate on addition of sodium hydroxide 
or sodium carbonate solution

\section{Potassium hexacyanoferrate (III)}
%\label{sec:}
Formula: \ce{K3Fe(CN)6}\\
Other name: Potassium ferricyanide\\
Description: yellow / orange salt\\
Use: confirmatory tests in qualitative analysis 
(makes an intense blue precipitate in the presence of iron (II) ions\\
Alternative: iron (II) ions will also instantly decolorize a weak, 
acidic solution of potassium manganate (VII)

\section{Potassium hydroxide}
%\label{sec:}
Formula: \ce{KOH}\\
Description: white crystals, 
deliquescent (poorly sealed containers may be just viscous water)\\
Use: volumetric analysis\\
Hazard: corrodes metal, 
burns skin, 
and can blind if it gets in eyes\\
Alternative: sodium hydroxide -- 
see \nameref{sec:commonsubs}.

\section{Potassium iodide}
\label{sec:potiodide}
Formula: \ce{KI}\\
Description: white crystals very similar in appearance to common salt, 
endothermic heat of solvation\\
Confirm: addition of weak potassium permanganate 
or bleach solution causes a clear KI solution to turn yellow/brown 
due to the formation of \ce{I2} (which then reacts with \ce{I-} to form soluble \ce{I3-})\\
Use: preparation of iodine solution for food tests in biology, 
preparation of iodine solutions for redox titrations, 
confirmatory test for lead in qualitative analysis\\
Local manufacture: Heat a pharmacy iodine tincture strongly until 
only clear crystals remain. 
In this process, 
the \ce{I2} will sublimate -- 
placing a cold dish above the iodine solution should cause must of the iodine 
to deposit as solid purple crystals. 
Note that the iodine vapors are harmful to inhale.
If you need \ce{KI} for a solution that may contain impurities, 
add ascorbic acid solution to dilute iodine tincture 
until the solution exactly decolorized.\\
Alternative (food tests): see \nameref{sec:iodinesol}\\
Alternative (redox titrations): 
often you can also use iodine solution for this; 
just calibrate the dilution of pharmacy tincture 
and the other reagents to create a useful titration\\
Alternative (qualitative analysis): 
confirm lead by the addition of dilute sulfuric acid -- 
white lead sulfate precipitates

\section{Potassium manganate (VII)}
%\label{sec:}
Formula: \ce{KMnO4}\\
Other names: potassium permanganate, 
permanganate\\
Description: purple/black crystals, 
sometimes with a yellow/brown glint, 
very soluble in water -- 
a few crystals will create a strongly purple colored solution\\
Hazard: powerful oxidizing agent -- 
may react violently with various compounds; 
solutions stain clothing (remove stains with ascorbic acid solution); 
crystals and concentrated solution discolor skin 
(the effect subsides after a few hours, 
but it is better to not touch the chemical!)\\
Use: strong oxidizer, 
self-indicating redox titrations, 
identification of various unknown compounds, 
diffusion experiments\\
Source: imported ``local'' medicine. 
Also sold in very small quantities in many pharmacies. 
May be available in larger quantities from hospitals.\\
Alternative (oxidizer): bleach (sodium hypochlorite), 
hydrogen peroxide\\
Alternative (diffusion experiments): solid or liquid food coloring, 
available in markets and small shops

\section{Potassium thiocyanate}
%\label{sec:}
Formula: \ce{KSCN}\\
Use: confirmation of iron (III) ions in qualitative analysis\\
Alternative: addition of sodium ethanoate 
should also produce a blood red solution; 
additionally, 
the test is unnecessary, 
as iron (III) ions is also the only chemical 
that will produce a red/brown precipitate 
with sodium hydroxide solution or sodium carbonate solution

\section{Propanone}
\label{sec:propanone}
Formula: \ce{H3CCOCH3}\\
Other names: acetone\\
Description: clear liquid miscible in water, 
smells like nail polish remover, 
quickly evaporates\\
Use: all-purpose lab solvent, 
iodoform reaction (kinetics, organic chemistry)\\
Hazard: highly flammable\\
Source: nail polish remover (mixture with ethyl ethanoate)\\
Alternative (volatile polar solvent): ethanol, 
including methylated spirits

\section{Silicon}
%\label{sec:}
Use: element\\
Source: fragments of broken solar panels; 
the cells are in part doped silicon

\section{Silicon dioxide}
%\label{sec:}
Description: clear solid\\
Source: quartz rock, 
quartz sand, 
glass

\section{Silver nitrate}
%\label{sec:}
Formula: \ce{AgNO3}\\
Description: white crystals, 
turn black if exposed to light (hence, 
the use of silver halides in photography)\\
Confirm: silvery-white precipitate formed with chlorides\\
Use: confirmatory test for chlorides in qualitative analysis\\
Hazard: poison, 
water pollutant\\
Alternative: heat sample together 
with a dilute solution of acidified potassium manganate (VII) -- 
decolorization confirms chlorides -- see \nameref{cha:qualana}

\section{Sodium}
%\label{sec:}
Description: very soft metal (cuts with a knife) 
with a silvery color usually obscured by a dull oxide; 
always stored under oil\\
Use: demonstration of reactive metals (add to water)\\
Hazard: reacts with air and violently with water. 
May cause fire.

\section{Sodium acetate}
%\label{sec:}
See \nameref{sec:sodiumeth}.

\section{Sodium carbonate}
%\label{sec:}
Formula: \ce{Na2CO3.10H2O} (hydrated), 
\ce{Na2CO3} (anhydrous)\\
Other names: soda ash, washing soda\\
Description: white powder completely soluble in water\\
Use: all-purpose cheap base, 
volumetric analysis, 
qualitative analysis, 
manufacture of other carbonates\\
Safety: rather caustic, keep off of hands and definitely out of eyes!\\
Source: commercial and industrial chemical supply -- 
should be very inexpensive\\
Local manufacture: dissolve sodium hydrogen carbonate in distilled water 
and boil for five or ten minutes 
to convert the hydrogen carbonate to carbonate. 
Let evaporate until crystals form. 
For volumetric analysis, 
the hydrated salt may always substitute 
for the anhydrous with a correction to the concentration -- 
see Chemical Substitutions for Volumetric Analysis

\section{Sodium chloride}
%\label{sec:}
Formula: \ce{NaCl}\\
Other names: common salt\\
Use: all-purpose cheap salt, 
qualitative analysis\\
Source: the highest quality salt in markets (white, 
finely powdered) is best. 
The iodine salts added to prevent goiter 
do not generally affect experimental results.

\section{Sodium citrate}
%\label{sec:}
Use: buffer solutions, 
preparation of Benedict's solution\\
Local manufacture: react sodium hydroxide 
and citric acid in a 3:1 ratio by mole\\
Alternative: to prepare Benedict's solution, 
see \nameref{sec:benedict}.

\section{Sodium ethanoate}
\label{sec:sodiumeth}
Formula: \ce{CH3CHOONa}\\
Other names: sodium acetate\\
Use: confirmation of iron (III) ions\\
Local manufacture: react sodium hydrogen carbonate 
and ethanoic acid in a 1:1 ratio by mole -- 
one 70~g box of baking soda to one liter of white vinegar labelled 5\%; 
if you need to err add excess sodium hydrogen carbonate. 
If the solid is required, 
leave to evaporate, 
but mostly likely you want the solution.

\section{Sodium hydrogen carbonate}
%\label{sec:}
Formula: \ce{NaHCO3}\\
Description: white powder, 
in theory completely soluble in cold water 
in practice often dissolves poorly\\
Other names: sodium bicarbonate, 
bicarbonate of soda\\
Use: all-purpose weak base, 
preparation of carbon dioxide, 
qualitative analysis\\
Source: small shops \\
Note: may contain ammonium hydrogen carbonate

\section{Sodium hydroxide}
%\label{sec:}
Formula: \ce{NaOH}\\
Other names: caustic soda\\
Description: white deliquescent crystals -- 
will look wet after a minute in contact with air 
and will fully dissolve after some time, 
depending on humidity and particle size\\ 
Use: all-purpose strong base, 
volumetric analysis, 
food tests in biology, 
qualitative analysis, 
preparation of sodium salts of weak acids\\
Hazard: corrodes metal, 
burns skin, 
and can blind if it gets in eyes\\
Source: industrial supply shops, 
supermarkets, 
hardware stores (drain cleaner)\\
Local manufacture: mix wood ashes in water, 
let settle, 
and decant; 
the resulting solution is mixed sodium and potassium hydroxides 
and carbonates and will work for practicing volumetric analysis\\
Note: ash extracts are about 0.1~M base and may be concentrated by boiling; 
this is dangerous, 
however, 
and industrial caustic soda is so inexpensive 
and so pure that there is little reason to use ash extract 
other than to show that ashes are alkaline 
and that sodium hydroxide is not exotic.

\section{Sodium hypochlorite solution}
%\label{sec:}
Formula: \ce{NaOCl_{(aq)}}\\
Other names: bleach\\
Local name: Jik
Use: oxidizing agent\\
Source: small shops, 
supermarkets\\
Local manufacture: electrolysis of concentrated salt water solution 
with inert (e.g. 
graphite) electrodes; 
4-5~V (three regular batteries) is best for maximum yield\\
Note: commercial bleach is usually 3.5\% sodium hypochlorite by weight

\section{Sodium nitrate}
%\label{sec:}
Formula: \ce{NaNO3}\\
Description: colorless crystals\\
Use: qualitative analysis\\
Hazard: oxidizer, 
used in the manufacture of explosives e.g. 
gunpowder\\
Alternative: to practice identification of the sodium cation, 
use sodium chloride\\
Local manufacture: Mix solutions of calcium ammonium nitrate 
and sodium carbonate and decant the clear solution 
once the precipitate (calcium carbonate) settles. 
Add a stoichiometric quantity of sodium hydroxide 
and let the reaction happen either outside 
or with under a condenser to trap the ammonia produced. 
The clear solution that remains should have no residual ammonia smell 
and should be neutral pH. 
Allow the solution to evaporate until sodium nitrate crystallizes.

\section{Sodium oxalate}
%\label{sec:}
Formula: \ce{Na2C2O4}\\
Use: demonstration of buffer solutions\\
Hazard: poisonous\\
Alternative: rather than oxalic acid / sodium oxalate, 
use citric acid / sodium citrate

\section{Sodium sulfate}
%\label{sec:}
Formula: \ce{Na2SO4}\\
Use: qualitative analysis\\
Local manufacture: combine precisely stoichiometric amounts 
of copper sulfate and sodium carbonate in distilled water. 
A balance is required to measure exactly the right amounts. 
Copper carbonate will precipitate and the resulting solution 
should contain only sodium sulfate. 
Filter out the copper carbonate and evaporate the clear solution to dryness. 
Sodium sulfate is thermally stable, 
so strong heating may be used to speed up evaporation.

\section{Sodium thiosulfate}
%\label{sec:}
Formula: \ce{Na2S2O3.5H2O}\\
Description: clear, 
hexagonal crystals\\
Use: reducing agent for redox titrations, 
sulfur precipitation kinetics experiments\\
Alternative (reducing agent): ascorbic acid\\
Alternative (kinetics): reaction between sodium hydrogen carbonate solution 
and dilute weak acid (citric acid or ethanoic acid), 
iodoform reaction (iodine solution and propanone)\\

\section{Succinic acid}
%\label{sec:}
Formula: \ce{HOOCCH2CH2COOH}\\
Description: white solid\\
Use: solute for partitioning in distribution (partition) experiments\\
Alternative: iodine also partitions well between aqueous and organic solvents; 
titrate iodine with ascorbic acid (or sodium thiosulfate) 
rather than sodium hydroxide as you would with succinic acid; 
ethanoic acid also partitions between some solvent combinations.

\section{Sucrose}
%\label{sec:}
Formula: \ce{C12H22O11}\\
Use: non-reducing sugar for food tests\\
Source: common sugar; 
the brown granular sugar at the market and in small shops is more common; 
the more refined white sugar is available in supermarkets\\
Note: sometimes impure sucrose causes Benedict's solution to turn green, 
even yellow. Try using more refined sugar.
Alternatively, insist to students than only a red/orange precipitate 
is a positive test for a reducing sugar during exams.

\section{Sudan III solution}
%\label{sec:}
Use: testing for fats in food tests\\
Alternative: ethanol-free iodine solution

\section{Sulfur}
%\label{sec:}
Local name: \textit{kibiriti upele}\\
Description: light yellow powder with distinct sulfurous smell\\
Use: element, 
preparation of iron sulfide\\
Source: large agricultural shops (fungicide, 
e.g. 
for dusting crops), 
imported ``local'' medicine

\section{Sulfuric acid}
%\label{sec:}
Formula: \ce{H2SO4}\\
Other names: battery acid\\
Local name: \textit{maji makali}\\
Description: clear liquid with increasing viscosity at higher concentrations; 
fully concentrated sulfuric acid ($\sim$18~M) is almost twice as dense as water 
and may take on a yellow, 
brown, 
or even black color from contamination\\
Use: all-purpose strong acid, 
volumetric analysis, 
qualitative analysis, 
preparation of hydrogen and various salts\\
Source: battery acid from petrol stations 
is about 4.5~M sulfuric acid and one of the least expensive sources of acid\\
Hazard: battery acid is dangerous; 
it will blind if it gets in eyes and will put holes in clothing. 
Fully concentrated sulfuric acid is monstrous, 
but fortunately never required. 
For qualitative analysis, 
``concentrated'' sulfuric acid means $\sim$5~M -- battery acid will suffice.\\
Note: ``dilute'' sulfuric acid should be about 1~M. 
To prepare this from battery acid, 
add one volume of battery acid to four volumes of water (e.g. 
100~mL battery acid + 400~mL water)

\section{Starch}
%\label{sec:}
Description: light weight, 
fine, 
white powder, 
not readily soluble in cold water\\
Confirm: makes a blue to black color with iodine solution\\
Use: preparation of starch solution\\
Source: supermarkets

\section{Starch solution}
%\label{sec:}
Use: sample for food tests, 
indicator for redox titrations involving iodine\\
Source: dilute the water left from boiling pasta or potatoes\\
Note: prepare freshly -- after a day or two it will start to rot!

\section{Tetrachloromethane}
\label{sec:tetrachloromethane}
Formula: \ce{CCl4}\\
Other names: carbon tetrachloride\\
Description: clean liquid, 
insoluble in and more dense than water\\
Use: organic solvent for distribution (partition) experiments\\
Hazard: toxic, 
probably carcinogen -- never use\\
Alternative: other organic solvents -- 
paint thinner and kerosene are the least expensive

\section{Trichloromethane}
\label{sec:trichloromethane}
Formula: \ce{CHCl3}\\
Other names: chloroform\\
Description: clear liquid, 
insoluble in and more dense than water, 
noxious smell\\
Use: rendering biological specimens unconscious prior to dissection, 
as an organic solvent for the distribution (partition) experiments\\
Alternative (biology): the specimen will die regardless 
so unless you are investigating the circulatory system 
you might as well kill it in advance; 
this also avoids the problem of specimens regaining consciousness 
before they bleed to death. 
See instructions in Dissections.\\
Alternative (chemistry): lower cost and safer organic solvents like kerosene can be used to practice distribution (partitioning), 
but unlike chloroform they are less dense than water.

\section{Tungsten}
%\label{sec:}
Symbol: \ce{W}\\
Use: element\\
Source: incandescent light bulb filaments\\
Extraction: wrap a light bulb in a rag and break it with a blunt object. 
The filament is the thin coiled wire. 
Dispose of the broken glass in a safe place, 
like a pit latrine.\\
Note: in a dead bulb, the cause of failure is probably the filament, 
so there might not be much left.

\section{Zinc}
\label{sec:zinc}
Description: firm silvery metal, 
usually coated with a dull oxide\\
Use: element, 
preparation of hydrogen, 
preparation of zinc carbonate and zinc sulfate\\
Source: dry cell batteries; 
under the outer steel shell is an inner cylinder of zinc. 
In new batteries, 
this whole shell may be extracted. 
In used batteries, 
the battery has consumed most of the zinc during the reaction, 
but there is generally an unused ring of zinc around the top 
that easily breaks off. 
Note that alkaline batteries, 
unlike dry cells, 
are unsafe to open -- and much more difficult besides.

\section{Zinc carbonate}
%\label{sec:}
Formula: \ce{ZnCO3}\\
Description: white powder\\
Use: qualitative analysis\\
Local manufacture: dissolve excess zinc metal 
in dilute sulfuric acid and leave overnight 
or until the acid is completely consumed. 
Decant the resulting zinc sulfate solution and 
mix with a sodium carbonate solution. 
Zinc carbonate will precipitate 
and may be purified by filtration and gentle drying.

\section{Zinc chloride and zinc nitrate}
%\label{sec:}
Description: clear, 
deliquescent crystals\\
Use: qualitative analysis\\
Alternative: to practice identification of zinc, 
use zinc sulfate or zinc carbonate; 
to practice identification of chloride use sodium chloride

\section{Zinc sulfate}
%\label{sec:}
Formula: \ce{ZnSO4}\\
Use: qualitative analysis\\
Local manufacture: dissolve excess zinc metal in dilute sulfuric acid 
and leave overnight or until the acid is completely consumed. 
Decant the resulting zinc sulfate solution and evaporate until crystals form.

\end{multicols}
