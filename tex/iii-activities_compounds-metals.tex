\section{Compounds of Metals}

\begin{multicols}{2}


\section*{Metal Oxides}


\subsection{Direct Preparation of a Metal Oxide}

%\begin{center}
%\includegraphics[width=0.4\textwidth]{./img/.png}
%\end{center}

\begin{description*}
%\item[Subtopic:]{}
\item[Materials:]{Copper wire, \nameref{sec:heatsources}, spoon}
\item[Setup:]{Clean the copper wire by scraping or burning off the insulation.}
\item[Procedure:]{Heat the copper wire strongly over a flame and observe any changes.}
%\item[Hazards:]{}
%\item[Questions:]{}
\item[Observations:]{Copper metal reacts with air to form black copper (II) oxide.}
%\item[Theory:]{}
%\item[Applications:]{}
\item[Notes:]{This experiment can also be tried with zinc metal, which turns yellow
when heated strongly in air. The yellow colour turns white when allowed to
cool. This product which is yellow when hot and white when cold is zinc
oxide (ZnO).}
\end{description*}

\subsection{Indirect Preparation of a Metal Oxide}

%\begin{center}
%\includegraphics[width=0.4\textwidth]{./img/.png}
%\end{center}

\begin{description*}
%\item[Subtopic:]{}
\item[Materials:]{Zinc metal, battery acid (5 M sulphuric acid), bottle, washing soda (sodium
carbonate), cloth, \nameref{sec:heatsources}, spoon}
\item[Setup:]{Prepare a solution of sodium carbonate by dissolving two tablespoons in
about 500 mL of water. Remove the zinc plate from an old dry cell.}
\item[Procedure:]{Put about 10 mL of sulphuric acid (battery acid) into a beaker. Add a small piece of zinc metal and allow it to dissolve. After the zinc has completely dissolved, add about 10 mL of sodium carbonate solution. Allow the precipitate to settle and then filter and collect it on the metal spoon. Heat the sample strongly on the spoon until a colour changed is noted.}
%\item[Hazards:]{}
%\item[Questions:]{}
\item[Observations:]{The precipitate changes from white to yellow when heated.}
\item[Theory:]{When zinc reacts with dilute sulphuric acid, a soluble zinc sulphate salt
forms by displacement reaction. When sodium carbonate
is added, zinc carbonate precipitate is formed (ZnCO$_3$). ZnCO$_3$ is white in
colour. When heated, the gas CO$_2$ is evolved and
the residue is ZnO. The ZnO is yellow when hot and white when cold.}
%\item[Applications:]{}
%\item[Notes:]{}
\end{description*}

%==================================================================================================%

\section*{Metal Hydroxides}


\subsection{Preparation of Metal Hydroxides}

%\begin{center}
%\includegraphics[width=0.4\textwidth]{./img/.jpg}
%\end{center}

\begin{description*}
%\item[Subtopic:]{}
\item[Materials:]{Steel wool, battery acid (5 M sulphuric acid), caustic soda (sodium hydroxide), cloth, funnel, bottles}
\item[Setup:]{Prepare a sodium hydroxide solution by adding 1 spoon of sodium hydroxide
to 100 mL of water.}
\item[Procedure:]{Add a small amount of steel wool to one container. Add about 10 mL of battery acid, adding more steel wool until all of the acid is consumed. Note the colour of the solution. 

When the reaction is finished, decant the contents of the bottle into a container of sodium hydroxide. A precipitate should form immediately. Observe the colour of the precipitate.

Pour the mixture with the precipitate into the filter funnel. Leave to
filter. Observe any change in colour. Once most of the liquid has passed through the filter, remove the solid from the filter funnel and leave to dry.}
%\item[Hazards:]{}
%\item[Questions:]{}
%\item[Observations:]{}
\item[Theory:]{The steel wool reacts with sulphuric acid to form iron (II) sulphate. This
solution reacts with sodium hydroxide solution to produce a green, gelatinous precipitate of iron (II) hydroxide. On exposure to air, this precipitate
oxidizes to reddish brown iron (III) hydroxide.}
%\item[Applications:]{}
%\item[Notes:]{}
\end{description*}

%==================================================================================================%

\section*{Carbonates and Hydrogen-Carbonates}

%==================================================================================================%

%\section*{Metal Nitrates}


\subsection{Preparation of Copper Carbonate}

%\begin{center}
%\includegraphics[width=0.4\textwidth]{./img/.jpg}
%\end{center}

\begin{description*}
%\item[Subtopic:]{}
\item[Materials:]{Epsom salt (magnesium sulphate) and/or copper (II) sulphate, washing
soda (sodium carbonate), funnel, cotton wool, beakers, spoons}
\item[Setup:]{Stuff cotton wool into the funnel to plug the hole at the bottom.}
\item[Procedure:]{In one beaker, add 2 spoons of magnesium sulphate or 1 spoon of
copper sulphate to about 100 mL of water and stir until dissolved.
In a second beaker, add 2 spoons of sodium carbonate to about 100
mL of water and stir until dissolved. 

Add the sodium carbonate solution to the magnesium sulphate / copper sulphate solution. A precipitate should form immediately. Filter the precipitate and allow it to dry.}
%\item[Hazards:]{}
%\item[Questions:]{}
%\item[Observations:]{}
\item[Theory:]{When magnesium sulphate solution is mixed with sodium carbonate solution, magnesium carbonate precipitates. When copper sulphate solution is
mixed with sodium carbonate solution, copper carbonate precipitates. This
demonstrates the preparation of metal carbonates by precipitation reactions.}
%\item[Applications:]{}
%\item[Notes:]{}
\end{description*}

\subsection{Preparation of Calcium Carbonate}

\begin{center}
\includegraphics[width=0.25\textwidth]{./img/source/limewater.jpg}
\end{center}

\begin{description*}
%\item[Subtopic:]{}
\item[Materials:]{Straw, lime water, bottle, test tube}
%\item[Setup:]{}
\item[Procedure:]{Add some lime water to a bottle or syringe. Blow into the solution through a straw.}
%\item[Hazards:]{}
%\item[Questions:]{}
\item[Observations:]{The solution turns a milky white colour.}
\item[Theory:]{Carbon dioxide blown from the straw will react with lime water Ca(OH)$_2$ to
form a white precipitate of CaCO$_3$. This shows that carbon dioxide reacts
with an alkali solution.}
\item[Applications:]{}
\item[Notes:]{}
\end{description*}

%==================================================================================================%

%\section*{Metal Chlorides}

%==================================================================================================%

\section*{Metal Sulphates}


\subsection{Preparation of Zinc Sulphate}

%\begin{center}
%\includegraphics[width=0.4\textwidth]{./img/.jpg}
%\end{center}

\begin{description*}
%\item[Subtopic:]{}
\item[Materials:]{Zinc metal, dilute sulphuric acid,bottles, evaporating dish, \nameref{sec:heatsources}, steel wool}
\item[Setup:]{Clean a zinc plate from an old dry cell using steel wool and cut it into small pieces.}
\item[Procedure:]{Add zinc pieces to a bottle followed by dilute sulphuric acid. After all the zinc has reacted, heat the solution on an evaporating plate until crystals are visible and collect the remains.}
%\item[Hazards:]{}
%\item[Questions:]{}
%\item[Observations:]{}
\item[Theory:]{Zinc reacts with the sulphuric acid and replaces hydrogen gas and form soluble zinc sulphate. The product formed from the evaporation of the zinc
sulphate solution is the white solid zinc sulphate.}
%\item[Applications:]{}
%\item[Notes:]{}
\end{description*}


\end{multicols}

\pagebreak