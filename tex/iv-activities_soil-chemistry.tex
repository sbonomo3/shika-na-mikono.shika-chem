\section{Soil Chemistry}

\begin{multicols}{2}


\section*{Soil Formation}


\subsection{Soil Formation}

%\begin{center}
%\includegraphics[width=0.4\textwidth]{./img/.png}
%\end{center}

\begin{description*}
%\item[Subtopic:]{}
\item[Materials:]{Dilute sulphuric acid, calcium carbonate rock (coral, limestone, marble) or egg shells, beaker}
%\item[Setup:]{}
\item[Procedure:]{Place carbonate rock or egg shells in a beaker. Add dilute sulphuric acid and observe.}
%\item[Hazards:]{}
%\item[Questions:]{}
\item[Observations:]{Bubbles of carbon dioxide should be observed where acid touches the rock. This shows that the acid is chemically reacting with the rock. Over time, the surface of the rock should also look corroded, more evidence of chemical weathering.}
\item[Theory:]{
\raggedright Soils are made by physical and chemical weathering of rocks. Chemical weathering is caused by the action of acids on carbonate rocks. These acids may be organic acids produced by soil organisms or carbonic acid from the dissolution of atmospheric carbon dioxide in water:\\
\begin{center}
\ce{ CO2_{(g)} <-> CO2_{(aq)} }
\end{center}
and then\\
\vspace{6pt}
\ce{ CO2_{(aq)} + H2O_{(l)} -> }\\
\ce{ H2CO3_{(aq)}	<-> H+_{(aq)} + HCO3^-_{(aq)} }\\
\vspace{6pt}
These acids react differently with different kinds of rocks. In the case of limestone, marble, coral, and other rocks made mostly from calcium carbonate, the reaction is:\\

\begin{center}
\ce{ H+_{(aq)} + CaCO3_{(s)} -> Ca^{2+}_{(aq)} + HCO3^-_{(aq)} }\\	
\end{center}

Note that the result is a solution of calcium hydrogen carbonate. This is the source of hard water.
Chemical weathering is a slow process. This activity speeds up the process by using dilute sulphuric acid so that students may more quickly see the result. Because sulphuric acid is a strong acid, it will also react with the hydrogen carbonate:\\
\begin{center}
\ce{ H+_{(aq)} + HCO3^-_{(aq)} -> H2O_{(l)} + CO2_{(g)} }\\	
\end{center}
The carbon dioxide thus produced can be observed as small bubbles forming on the surface of the rock. Note that in real chemical weathering, carbon dioxide is generally not produced, and instead the reaction stops with a solution of calcium hydrogen carbonate.}
%\item[Applications:]{}
\item[Notes:]{Rocks with more complicated chemical composition are also subject to chemical weathering.}
\end{description*}

\vfill
\columnbreak

\subsection{Cement Making and Erosion}

%\begin{center}
%\includegraphics[width=0.4\textwidth]{./img/.png}
%\end{center}

\begin{description*}
%\item[Subtopic:]{}
\item[Materials:]{Cement, sand, water, plastic water bottles, large plastic container}
\item[Setup:]{Add equal volumes of cement and sand to a large plastic container or wheelbarrow (the actual volume used is not important). Add water to make a paste and pour off into a plastic water bottle with the top cut off. This bottle acts as a mold for the cement. Then, add a second volume of sand to the large plastic container. Pour into a plastic water bottle. Add a third volume of sand, and then pour into a water bottle. Repeat this procedure until you have added 12 volumes of sand. Let the cement dry overnight and cut off the plastic water bottle. Label and keep each different sample of cement.}
\item[Procedure:]{Take the cement pieces and place them outside to bear the elements. Record their status each week.}
%\item[Hazards:]{}
%\item[Questions:]{}
%\item[Observations:]{}
\item[Theory:]{The ratio of cement to sand decreases through each dilution. This means that the strength that holds the cement together decreases as the ratio of cement to sand increases. We can see this by leaving the different pieces of cement outside to erode. The strongest pieces will resist erosion the most. Those that have a 1:10 ratio of cement to sand will erode very easily.}
\item[Applications:]{This is why most cement blocks look like they are melting when it rains. The cement is too diluted to resist erosion effectively. Over the course of a year, the cement that has a 1:10 or 1:12 ratio will erode while the other pieces of cement will not erode.}
%\item[Notes:]{}
\end{description*}

%\vfill
%\columnbreak

%==================================================================================================%

\section*{Soil Nutrients}


\subsection{Leaching}

%\begin{center}
%\includegraphics[width=0.4\textwidth]{./img/.png}
%\end{center}

\begin{description*}
%\item[Subtopic:]{}
\item[Materials:]{Sand, solid food colouring, filter funnel, beaker, water}
\item[Setup:]{Prepare a mixture of sand and solid food colouring.}
\item[Procedure:]{Emphasize that the food colouring represents soluble minerals and soil nutrients. Place the sand mixture in a filter funnel placed over a beaker. Add water and observe the colour of the filtrate.}
%\item[Hazards:]{}
%\item[Questions:]{}
\item[Observations:]{The filtrate takes on the colour of the sand that it passes through.}
\item[Theory:]{Chemicals present can often pass through the soil as water makes its way through the soil and end up in water sources.}
%\item[Applications:]{}
%\item[Notes:]{}
\end{description*}


%==================================================================================================%

%\section*{Manures and Fertilizers}

%==================================================================================================%

\section*{Soil Reaction}

\subsection{Measuring Soil pH}

%\begin{center}
%\includegraphics[width=0.4\textwidth]{./img/.png}
%\end{center}

\begin{description*}
%\item[Subtopic:]{}
\item[Materials:]{Various soil samples, rosella leaves, paper, bottles, water}
\item[Setup:]{Prepare an indicator solution or litmus paper by adding rosella leaves to hot water and dipping thin strips of paper into the solution.}
\item[Procedure:]{Put soil in a bottle. Add water to the soil and stir. Test the liquid with indicating paper. Record any changes.}
%\item[Hazards:]{}
%\item[Questions:]{}
%\item[Observations:]{}
\item[Theory:]{Some soils are neutral in pH. Others are acidic or basic, depending on the composition of the soil. This activity is meant to demonstrate the existence of acidic and basic soils. }
%\item[Applications:]{}
\item[Notes:]{Traditionally, this activity is performed with universal indicator. However, exceptionally acidic or basic soils should be possible to detect using red and blue indicating paper, which may be locally made.}
\end{description*}

\vfill
\columnbreak

\subsection{Raising Soil pH by Liming}

%\begin{center}
%\includegraphics[width=0.4\textwidth]{./img/.png}
%\end{center}

\begin{description*}
%\item[Subtopic:]{}
\item[Materials:]{An acidic soil (as determined from above activity), lime (cement or calcium hydroxide), water, indicator paper}
%\item[Setup:]{}
\item[Procedure:]{Add lime to an acidic soil sample. Test again with indicator paper.}
%\item[Hazards:]{}
%\item[Questions:]{}
%\item[Observations:]{}
\item[Theory:]{The addition of lime will cause the soil to become more basic, and thus have a higher pH.}
%\item[Applications:]{}
%\item[Notes:]{}
\end{description*}

\subsection[Lowering Soil pH with Ammonium Sulphate]{Lowering Soil pH with \hfill \\ Ammonium Sulphate}

%\begin{center}
%\includegraphics[width=0.4\textwidth]{./img/.png}
%\end{center}

\begin{description*}
%\item[Subtopic:]{}
\item[Materials:]{A basic soil (as determined from above activity), ammonium sulphate fertilizer, water,  indicating paper}
%\item[Setup:]{}
\item[Procedure:]{Add ammonium sulphate to a basic soil sample from the soil pH activity. Test again with indicator paper.}
%\item[Hazards:]{}
%\item[Questions:]{}
%\item[Observations:]{}
\item[Theory:]{The addition of ammonium phosphate will cause the soil to become more acidic, and thus have a lower pH.}
%\item[Applications:]{}
%\item[Notes:]{}
\end{description*}


\end{multicols}

\pagebreak